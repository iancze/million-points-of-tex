% \documentclass[twocolumn]{aastex62}
\documentclass[modern]{aastex62}

\usepackage{bm}

\newcommand{\radmc}{\texttt{RADMC-3D}}
\newcommand\kms{\ifmmode{\rm km\thinspace s^{-1}}\else km\thinspace s$^{-1}$\fi}
\newcommand{\todo}[1]{ \textcolor{red}{#1}}
\newcommand{\vt}{ {\bm \theta}}
\newcommand{\msun}{M$_\odot$}
\newcommand{\obj}{GW\,Ori}
\newcommand{\twelve}{${}^{12}$CO}
\newcommand{\thirteen}{${}^{13}$CO}
\newcommand{\eighteen}{C${}^{18}$O}

\newcommand{\vd}{\boldsymbol{\mathcal{V}}} % visibility data 
\newcommand{\vm}{\boldsymbol{\mathcal{W}}} % visibility model
\newcommand{\btheta}{\boldsymbol{\theta}} % parameter vector

% standardizing notation
%
% disk parameters
% i_\mathrm{disk} (not i_d or i)
%
% orbit parameters
% i_\mathrm{in}
% i_\mathrm{out}

% stellar masses
% M_\ast 

\begin{document}


\title{A Million Points of Light}

\correspondingauthor{Ian Czekala}
\email{iczekala@berkeley.edu}

\author[0000-0002-1483-8811]{Ian Czekala}
\altaffiliation{NASA Hubble Fellowship Program Sagan Fellow}
\affiliation{Department of Astronomy, 501 Campbell Hall, University of California, Berkeley, CA 94720-3411, USA}


\begin{abstract}
Fast computation of protoplanetary disk channel maps for molecular gas lines. With gradients (we hope).
\end{abstract}

\keywords{protoplanetary disks -- stars: pre-main sequence}

\section{Introduction} \label{sec:intro}

There have been some cool new techniques to model rotation curves derived from high resolution (spatial and spectra) images of protoplanetary disks. for example, \citet{yen16,yen18} stacking lines to detect faint transitions and measure dynamical masses. Also to probe perturbations to velocity fields to search for the influence of exoplanets \citet{teague18a} and perturbations to the pressure profiles in disks \citet{teague18c}. 

In this paper, we try to incorporate some ideas inspired by these data-driven techniques into our modus operandi, which is fitting the visibilities directly using the Fourier transform of a sky brightness model. We'll also try to be smart about how we calculate the model visibilities, in such a way that we can also calculate derivatives in order to efficiently use high dimensional optimization and inference schemes. 

Because our goal is to make the Fourier part somewhat generic to the parameterization of the sky model, we'll take the following order. First, we'll describe just enough of the sky model to give context to what we mean. Then, we'll develop the mathematical framework to calculate the model visibilities and their derivatives. Finally, we'll circle back and describe a specific implementation for the sky model and discuss how it might be extended.

% \begin{figure*}[ht!]
% \begin{center}
%   \includegraphics[width=\linewidth]{moments.pdf}
%   \figcaption{({\it left}) A 226\,GHz continuum image.  Contours start at 5$\times$ the RMS noise level and increase by factors of 2.  The synthesized beam geometry is shown in the lower left corner.  ({\it middle, left to right}) Maps of the $^{12}$CO, $^{13}$CO, and C$^{18}$O velocity-integrated intensities (contours, starting at 10, 3, and 3$\times$ the RMS noise levels, respectively, and increasing by factors of 2) overlaid on the intensity-weighted projected velocities (color-scale).  Note the prominent molecular cloud contamination in the $^{12}$CO map (see also Fig.~\ref{fig:chanmaps}).  ({\it right}) Spatially integrated spectra (inside the same {\tt CLEAN} mask, and smoothed with an 0.85\,km\,s$^{-1}$ Hanning kernel) for each CO line.
%   \label{fig:moments}}
%   \end{center}
% \end{figure*}

\section{Sky model basics}
Interferometers like ALMA measure the Fourier transform of the sky brightness, as a function of frequency. Our first task is to describe the sky brightness model, $I(x, y, \nu)$, which is a function of position and frequency. This has units of intensity, or something like Jy/beam or Jy/sterradian. The position x, y, is directly related to the RA, DEC, based upon the distance to the source, so we end up with a model of $I(l, m, \nu)$, where $l$ is the RA coordinate (in radians) and $m$ is the dec coordinate (in radians). Like RA, $l$ increases towards the east (left) of images.

The model parameters generally break down into two groups, those that are required to describe the morphology of the intensity, and those geometric parameters that describe the rotation or translation of the emission. The second group of parameters can be easily applied in the Fourier domain via the shift and rotation theorems.

\subsection{Likelihood Calculation}

Define our visibility data as $\boldsymbol{\mathcal{V}}$.
Define our visibility model as $\boldsymbol{\mathcal{W}}(\boldsymbol{\theta})$.

Typically done as 
\begin{equation}
    {\cal L}(\boldsymbol{\theta}) = {\cal N}(\boldsymbol{\mathcal{V}} |\, \boldsymbol{\mathcal{W}}(\boldsymbol{\theta}), \boldsymbol{C})
\end{equation}
Typically we just treat the visibilities as independent measurements, and so we have a simple $\chi^2$ likelihood of 
\begin{equation}
    \ln {\cal L}(\btheta) = - \frac{1}{2} \sum_i^N \frac{|\vd - \vm(\btheta)|^2}{\sigma_i^2}.
\end{equation}
Note that the likelihood is a scalar quantity. Next, we want to calculate the derivative of the likelihood
\begin{equation}
    \frac{\mathrm{d}\mathcal{L}(\btheta)}{\mathrm{d}\btheta}.
\end{equation}
The main image is $I(x,y,\nu)$. The 2D Fourier transform (FT) of this image is 
\begin{equation}
    V(u,v,\nu) = \mathcal{F}(I).
\end{equation}
We're assuming that each frequency channel of the channel maps are independent. Baseline-dependent phase errors may make this less true. See \citet{hezaveh13} for the proper way to bake these into the inference loop, if we so desire, but we will just ignore them for now.

Real $\Re$ and imaginary $\Im$ and $\imath$.

DFM points out that, in theory, we can also create disk code that can deliver us a derivative of the likelihood function. This is great, because we can feed this into techniques like Hamiltonian Monte Carlo and access a much larger set of parameter space. We are going to need this, I think, if we actually want disk models that are realistic.

The idea is to start writing the change in derivative with respect to the interpolated visibilities.

Then, we probably need to write down the change in interpolated visibility. Due to the FFT, though, this is going to require a knowing the changes of all of the other pixels. 

If we can reduce this to knowing the change in a single pixel, then that change in intensity, with respect to the parameters, is possible to compute.

It may be possible to introduce correction to the phase errors, as well.

\section{Modeling the radial intensity profile}
How smooth is the radial profile, and can we use a Nuker profile to model it?


\acknowledgments

% \software{CASA \citep[v4.4;][]{mcmullin07}, IRAF \citep{tody86,tody93}, DiskJockey \citep{czekala15a}, RADMC-3D \citep{dullemond12}, emcee \citep{foreman-mackey13}, VARTOOLS \citep{hartman16}, Astropy \citep{astropy13}}

\bibliographystyle{yahapj.bst}
\bibliography{biblio.bib}


\end{document}
