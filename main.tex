% \documentclass[twocolumn]{aastex62}
\documentclass[modern]{aastex62}

\usepackage{bm}
\usepackage{amsmath}

\usepackage[OT2,T1]{fontenc}
\DeclareSymbolFont{cyrletters}{OT2}{wncyr}{m}{n}
\DeclareMathSymbol{\Sha}{\mathalpha}{cyrletters}{"58}


\newcommand{\radmc}{\texttt{RADMC-3D}}
\newcommand\kms{\ifmmode{\rm km\thinspace s^{-1}}\else km\thinspace s$^{-1}$\fi}
\newcommand{\todo}[1]{ \textcolor{red}{#1}}
\newcommand{\vt}{ {\bm \theta}}
\newcommand{\msun}{M$_\odot$}
\newcommand{\obj}{GW\,Ori}
\newcommand{\twelve}{${}^{12}$CO}
\newcommand{\thirteen}{${}^{13}$CO}
\newcommand{\eighteen}{C${}^{18}$O}

\newcommand{\im}{\boldsymbol{I}_\blacksquare}
\newcommand{\vd}{\boldsymbol{\mathcal{V}}} % visibility data
\newcommand{\vm}{\boldsymbol{\mathcal{W}}} % visibility model
\newcommand{\bbeta}{\boldsymbol{\beta}} % disk structure parameter vector
\newcommand{\btheta}{\boldsymbol{\theta}} % all parameter vector
\newcommand{\ct}{c_{\,\blacksquare}}

% standardizing notation
%
% disk parameters
% i_\mathrm{disk} (not i_d or i)
%
% orbit parameters
% i_\mathrm{in}
% i_\mathrm{out}

% stellar masses
% M_\ast

\begin{document}


\title{A Million Points of Light}

\correspondingauthor{Ian Czekala}
\email{iczekala@berkeley.edu}

\author[0000-0002-1483-8811]{Ian Czekala}
\altaffiliation{NASA Hubble Fellowship Program Sagan Fellow}
\affiliation{Department of Astronomy, 501 Campbell Hall, University of California, Berkeley, CA 94720-3411, USA}


\begin{abstract}
Fast computation of protoplanetary disk models of molecular gas emission. With likelihood gradients (we hope).
\end{abstract}

\keywords{protoplanetary disks -- stars: pre-main sequence}

\section{Introduction} \label{sec:intro}

There have been some cool new techniques to model rotation curves derived from high resolution (spatial and spectra) images of protoplanetary disks. For example, \citet{yen16,yen18} have stacked lines to detect faint transitions and measure dynamical masses. Extensions to these techniques have also been used to probe perturbations to velocity fields to search for the influence of exoplanets \citep{teague18a} and perturbations to the pressure profiles in disks \citep{teague18c}.

In this paper, we try to incorporate some inspirations from these data-driven techniques into our typical modus operandi, which is fitting the visibilities directly using the Fourier transform of a sky brightness model. We'll also try to be smart about how we calculate the model visibilities so that we can also calculate model derivatives in order to efficiently use high dimensional optimization and inference schemes. Because our goal is to make the derivatives part generic to the exact parameterization of the sky model, we'll be explaining things in the following order. First, we'll describe just enough of the sky model in order to give some context. Then, in gory detail we'll develop the mathematical framework to calculate the model visibilities and their derivatives from a generic sky model. Finally, we'll circle back and describe a specific implementation for the sky model and discuss how it might be extended to do cool stuff.

% \begin{figure*}[ht!]
% \begin{center}
%   \includegraphics[width=\linewidth]{moments.pdf}
%   \figcaption{({\it left}) A 226\,GHz continuum image.  Contours start at 5$\times$ the RMS noise level and increase by factors of 2.  The synthesized beam geometry is shown in the lower left corner.  ({\it middle, left to right}) Maps of the $^{12}$CO, $^{13}$CO, and C$^{18}$O velocity-integrated intensities (contours, starting at 10, 3, and 3$\times$ the RMS noise levels, respectively, and increasing by factors of 2) overlaid on the intensity-weighted projected velocities (color-scale).  Note the prominent molecular cloud contamination in the $^{12}$CO map (see also Fig.~\ref{fig:chanmaps}).  ({\it right}) Spatially integrated spectra (inside the same {\tt CLEAN} mask, and smoothed with an 0.85\,km\,s$^{-1}$ Hanning kernel) for each CO line.
%   \label{fig:moments}}
%   \end{center}
% \end{figure*}

\section{Sky model basics}
Our first task is to describe the sky brightness model, which is a function of sky position and, for molecular line datasets, also frequency
\begin{equation}
    I(\alpha, \delta, \nu) \quad [\mathrm{Jy/beam}].
\end{equation}
The sky brightness has units of intensity, which usually means something like Jy/beam or Jy/steradian. R.A. increases towards the east (left). We'll get into this in more detail later, but suppose we have a setup that generates a sky brightness model (i.e., images) corresponding to specific protoplanetary disk parameters. The model parameters generally break down into two groups: those that are required to describe the morphology of the intensity (things like size of the disk, strength of the gravity field, etc\ldots), and those geometric parameters that describe the rigid-body positioning of the brightness model, i.e. translation and rotation. We'll lump the first group of parameters into $\bbeta$. Currently, $\bbeta$ contains about 7 - 10 parameters for simple disk models, though realistically we'd like to jack this up to 20 or more if we can, since current ALMA datasets demand this level of detail. The second group of parameters includes the R.A. ($\delta_\alpha$) and Dec ($\delta_\delta$) offsets of the disk centroid from the phase center of the observations and the position angle of the ascending note $\Omega$; their effects on the model can be easily applied in the Fourier domain via the shift and rotation theorems. Together, we'll call all of the  parameters describing the sky brightness model $\btheta = \left \{ \bbeta, \delta_\alpha, \delta_\delta, \Omega \right \}$. This should be enough information to get started describing how to calculate the model visibilities and their derivatives. We'll come back to describing how to calculate $I(\alpha, \delta, \nu |\, \bbeta)$ in more detail in a later section.  

\section{Calculating visibilities}
Interferometers like ALMA measure the Fourier transform of the sky brightness, called the visibility function $V$
\begin{equation}
  V(u, v, \nu) \rightleftharpoons I(\alpha, \delta, \nu).
\end{equation}
This function is spanned by the $u$ and $v$ coordinates called \emph{spatial frequencies}. If $\alpha$ and $\delta$ are in units of radians, then $u$ and $v$ have units of $\lambda$, the wavelength of light of the observation.\footnote{Some textbooks also use the direction cosines $l$ and $m$ instead of R.A. and Dec to be more explicit about the radian thing, and there are some technicalities in more complicated imaging situations. For our purposes, these are equivalent, so we'll just use $\alpha$ and $\delta$ to avoid introducing yet more symbols.} The visibility function is complex-valued. A typical measurement set from ALMA consists of discrete measurements of $V$ at many tens of thousands if not hundreds of thousands of $\{ u, v\}_i$ points, across a range of frequencies. We'll call the collection of these measurements $\vd$ the \emph{measurement set}, or simply the \emph{visibilities}.

Although there are some interesting projects to make models directly in the Fourier domain, for now we're going to stick with creating a discrete sky brightness model in the image domain (i.e., a 256 $\times$ 256 $\times N_\mathrm{chan}$ image cube, frequently called the ``channel maps'') and then using the FFT to calculate discrete samples in the Fourier domain. For the sake of getting everything down in math, we'll start by treating a single frequency channel at a time. 

To make the appeal to signal processing explicit, we will call an actual image array $\im$, which is a sampling of the continuous image $I(\alpha, \delta)$ at specific $\left \{\alpha, \delta \right \}_i$ pixels. We can represent this as 
\begin{equation}
    \im = \Sha(\alpha/\Delta \alpha, \delta / \Delta \delta) \cdot I(\alpha, \delta)
\end{equation}
where $\Delta \alpha$ and $\Delta \delta$ are the sizes of the pixels in radians, and $\Sha$ is Bracewell's Sha function\footnote{You see why the choice of $\alpha$ and $\delta$ for R.A. and Dec is not totally ideal here, but hopefully it's clear where I mean declination and where I mean a $\delta$-function.}
\begin{equation}
    \Sha(\alpha/\Delta \alpha, \delta / \Delta \delta)  = \sum_j^\infty \sum_k^\infty \delta(j - \alpha/\Delta \alpha, k - \delta/\Delta \delta).
\end{equation}
For terminology's sake, we'll call the output of the FFT the \emph{dense visibility model}, or $\vm_\blacksquare$
\begin{equation}
    \vm_\blacksquare = \mathfrak{F} \im.
\end{equation}
The Fourier transform of the Sha function is itself, so, via the similarity Fourier theorem,
\begin{equation}
    \Sha(\alpha/\Delta \alpha, \delta / \Delta \delta) \rightleftharpoons \Delta \alpha \Delta \delta \Sha(\Delta \alpha u, \Delta \delta v).
\end{equation}
We also have 
\begin{equation}
    \vm_\blacksquare = \Delta \alpha \Delta \delta \Sha(\Delta \alpha u, \Delta \delta v) \cdot V(u, v).
\end{equation}
In order to create a set of \emph{model visibilities} $\vm$ corresponding to the exact values of $\{ u, v\}_i$ in the dataset, we'll need to do some sort of interpolation from the regularly spaced samples in $\vm_\blacksquare$. 

\subsection{Interpolation schemes}
While in theory one could just use a bilinear interpolation scheme to go from $\vm_\blacksquare$ to $\vm$, this isn't the wisest thing to do. That's because any interpolation scheme with finite support will in some way affect the signal that we have sampled, and we want to control these influences as best we can. The accepted best practice among radio astronomers is to write the interpolation as the convolution of the discrete $\vm_\blacksquare$ with a prolate spheroidal wavefunction $C$, so as to form a continuous function, which is then resampled at the $\{ u, v\}_i$ points corresponding to $\vd$. This is described in detail in \citet[ch 7.3,][]{synthesis99} and \citet{schwab84}, but from the inverse perspective of putting irregularly spaced samples \emph{back onto} a regularly spaced grid to do an FFT and get an image. We've always implemented this as a pretty basic \texttt{for} loop in our codes, but let's try to write this in matrix form because that should help clarify how $\vm$ is linear w.r.t to $I$ (I think it is) and how we're going to take derivatives.

Let the sampling function $S$ be a bunch of $\delta$-functions at the locations of the $\{ u, v\}_i$ points in the dataset (not regularly spaced),
\begin{equation}
    S(u,v) = \sum_i^N \delta(u - u_i, v - v_i)
\end{equation}
where $N$ is the number of visibilities in the dataset. To generate $\vm$, we could imagine doing the following 
\begin{eqnarray}
    \vm &=& [\mathfrak{F} \im * C] \cdot S \quad \mathrm{\color{red}{WRONG}} \label{eqn:wrong-conv}\\
    \vm &=& [\vm_\blacksquare * C] \cdot S \quad \mathrm{\color{red}{WRONG}}
\end{eqnarray}
where $C$ is some convolutional kernel with finite support and $*$ is the convolution operator. However, if we rearrange things using the Fourier convolution/multiplication theorem, then this is equivalent to 
\begin{eqnarray}
\vm &=& \mathfrak{F}[\im \cdot \mathfrak{F}C] \cdot S \quad \mathrm{\color{red}{WRONG}}\\
\vm &=& \mathfrak{F}[\im \cdot c] \cdot S \quad \mathrm{\color{red}{WRONG}}
\end{eqnarray}
and we see that the choice of convolution kernel $C$ is leaving some effect $c$ on the original signal in our image, where $c \rightleftharpoons C$. To remedy this, we can pre-divide the image by $c$ in order to achieve an unaffected image at the end. Let's write the pre-division of the image as 
\begin{equation}
    \im^c = \im \circ \frac{1}{\ct}.
\end{equation}
where we use the $\circ$ symbol to signify element-by-element array multiplication, also called an entrywise product, Hadamard product, or Schur product. This leads to 
\begin{eqnarray}
\vm &=& \mathfrak{F} \left [ \im^c \cdot \mathfrak{F}C \right ] \cdot S \\
\vm &=& \mathfrak{F}\left [\frac{\im}{c} \cdot c \right ] \cdot S \\
\vm &=& \mathfrak{F} \im \cdot S \quad \mathrm{\color{green}{CORRECT}}.
\end{eqnarray}
Therefore, as opposed to Eqn~\ref{eqn:wrong-conv}, the convolution and sampling operation we actually want to carry out is 
\begin{equation}
    \vm = [\mathfrak{F} \im^c * C] \cdot S \quad \mathrm{\color{green}{CORRECT}}
\end{equation}

Now that we've covered the basics of the signal processing, let's write the generation of $\vm$ from $\im$ in true matrix form. To make the calculation of $C$ and $c$ explicit, we'll first need to define the prolate spheroidal function $\psi(\eta)$.\footnote{There are many variant forms of $\psi$ depending on how many nearest neighbor gridpoints we'd like to include (the support size $m$) and how concentrated we'd like the peak to be (the exponent $\alpha$), but the standard one employed by radio astronomers seems to be the spheroid with $m = 6$ and $\alpha = 1$, so from here on assume that we're referring to that function so that we can drop the extra subscripts. If you're interested in a different kernel choice, all of the necessary detail is in \citet{schwab84}.} This function is usually implemented via a polynomial approximation like many other special functions, which we'll describe in the appendix. For a mental picture, it's essentially like a Gaussian taper from 1 to 0 over the domain [0,1]. We'll use it to create a two dimensional kernel that convolves over a $6 \times 6$ subgrid of points within $\vm_\blacksquare$.

The two dimensional convolution kernel is separable, so we have $C(u,v) = C_1(u)C_2(v)$. The one dimensional convolutional kernel is 
\begin{equation}
    C(\eta) = |1 - \eta^2| \psi(\eta).
\end{equation}
where $\eta(u) = \frac{u}{3 \Delta u}$, $\eta(v) = \frac{v}{3 \Delta v}$; and $\Delta u$ and $\Delta v$ are the spacings between points in $\vm_\blacksquare$. These spacings are determined by the size of $\im$ and the spacings in $\Delta \alpha$ and $\Delta \delta$. If the image size is $N_\alpha \times N_\delta$, then
\begin{eqnarray}
    \Delta u &= \frac{1}{N_\alpha \Delta \alpha}\\
    \Delta v &= \frac{1}{N_\delta \Delta \delta}.
\end{eqnarray}
For completeness, the full two dimensional kernel is
\begin{equation}
    C(u,v) = |1 - \eta(u)^2| |1 - \eta(v)^2| \psi(\eta(u)) \psi(\eta(v)) 
\end{equation}
The one dimensional taper in the map plane is simply 
\begin{equation}
    c(\eta) = \psi(\eta)
\end{equation}
where $\eta(\alpha) = 2 \alpha \Delta u$ and  $\eta(\delta) = 2 \delta \Delta v$, e.g., $\eta$ is 0 in the center of the image and 1 at the edges. As noted by \citet{synthesis99}, $\psi$ is a ``finite Fourier self-transform,'' which means if you truncate it to the window [-1,1], normalize it appropriately, and Fourier transform it, you get back $\psi$. For completeness, the full map plane taper is 
\begin{equation}
    c(\alpha, \delta) = \psi(\eta(\alpha)) \psi(\eta(\delta)).
\end{equation}
Calculating the map plane taper $\ct$ is a one-time operation set by the pixel sizes and image extent. 

Now, let's write the FFT of the taper-corrected image as 
\begin{equation}
    \vm^c = \mathfrak{F} \im^c
\end{equation}
To make the interpolation/convolution process a bit more tractable, let's think about interpolating $\vm^c_\blacksquare$ to a single $\{u, v\}_i$ point. This is done by evaluating $C(u, v)$ at all of the gridpoints that fall within the finite support, multiplying them by the value of $\mathcal{W}^c_{\blacksquare jk}$ at that gridpoint, summing them, and then dividing by the normalization. Our choice of kernel and the fact that we're on a regular grid means that this operation will always involve the 36 nearest gridpoints
\begin{equation}
    V(u_i, v_i) = \frac{\sum_j^6 \sum_k^6 C(u_j - u_i, v_k - v_i) \mathcal{W}^c_{\blacksquare jk}}{\sum_j^6 \sum_k^6 C(u_j - u_i, v_k - v_i)}.
\end{equation}
Thankfully, because neither the $u$ and $v$ points in the dense visibility model nor the $\{u,v\}_i$ points in the measurement set change with model parameters, calculating the convolutional prefactors is also a one-time process. We can also precalculate the denominator $w_i = \sum_j^6 \sum_k^6 C(u_j - u_i, v_k - v_i)$ and turn this into a single loop
\begin{equation}
 V(u_i, v_i) = \sum_l^{36} C(u_l - u_i, v_l - v_i) \mathcal{W}^c_{\blacksquare l} / w_i.
 \end{equation}
 where the index $l$ covers all of the combinations previously in the double sum over $\{j, k\}$. 
 
 Next up is writing this operation using as convolutional matrix. Technically, $\vm^c_\blacksquare$ doesn't need to be in a 2D matrix format, so to make the matrix math work, we can ravel it into a vector with length ($N_\alpha \times N_\delta$, 1), which we'll call $\hat{\vm_\blacksquare^c}$. We'll call the row vector of pre-evalutions of the kernel $\hat{\boldsymbol{C}}_i = C(u_l - u_i, v_l - v_i) / w_i$, which has shape (1, $N_\alpha \times N_\delta$). Most of the columns of this vector will be $0$, except those 36 nearest to the point $u_i, v_i$. Then, we can write $V(u_i, v_i)$ as a matrix product
 \begin{equation}
     V(u_i, v_i) = \hat{\boldsymbol{C}}_i \hat{\vm_\blacksquare^c}.
 \end{equation}
 To get the full measurement set, we need to construct a $\hat{\boldsymbol{C}}$ with size ($N_\mathrm{vis}$, $N_\alpha \times N_\delta$), filled with the proper kernel pre-evaluations corresponding to each $\{u,v\}_i$ point in the measurement set. As you might imagine, $\hat{\boldsymbol{C}}$ will be a very sparse matrix. Then we have
  \begin{equation}
     \vm = \hat{\boldsymbol{C}} \hat{\vm_\blacksquare^c}.
 \end{equation}

\subsection{Transforming visibilities with theorems}
The position offsets from the phase center of the observations, $\delta_\alpha$ and $\delta_\delta$, transform the model visibilities in straightforward, analytic ways. The two dimensional shift theorem says that a translational offset in the image plane results in a phase shift in the Fourier plane \citep{bracewell00}
\begin{equation}
I(\alpha - \delta_\alpha, \delta - \delta_\delta) \rightleftharpoons \exp \left \{ -2 \pi \imath (\delta_\alpha u + \delta_\delta v) \right \} V(u, v).
\end{equation}
This means that every model visibility is phase-shifted by an amount that depends on its spatial frequencies $\{u_i, v_i\}$. Because of the $\exp$, this is a non-linear operation, but simple to implement.

If the sky image is rotated by an angle $\Omega$ such that the positions $\alpha$ and $\delta$ become
\begin{eqnarray}
\alpha^\prime = \alpha \cos \Omega - \delta \sin \Omega \label{eqn:rota}\\
\delta^\prime = \alpha \sin \Omega + \delta \cos \Omega \label{eqn:rotd}
\end{eqnarray}
then its Fourier transform is rotated by the same angle and in the same sense \citep{bracewell00}
\begin{eqnarray}
u^\prime = u \cos \Omega - v \sin \Omega \label{eqn:rotu}\\
v^\prime = u \sin \Omega + v \cos \Omega. \label{eqn:rotv}
\end{eqnarray}
Since this operation requires resampling the visibility array at the rotated spatial frequencies, unlike the translational offset, it must be done using $\vm_\blacksquare$ rather than $\vm$. This operation involves recalculating $\hat{\boldsymbol{C}}$. So, unfortunately neither the translational offsets nor the rotation are simple linear operations applied to the model visibilities, but they are useful to keep in mind because it means we can make these model changes in the Fourier domain without needing to recalculate $\im$ or even redo the FFT that delivers $\vm_\blacksquare$.

\subsection{Likelihood Calculation}

With model visibilities in hand, we can calculate the likelihood of the dataset via
\begin{equation}
    {\cal L}(\vd |\, \boldsymbol{\theta}) = {\cal N}(\boldsymbol{\mathcal{V}} |\, \boldsymbol{\mathcal{W}}(\boldsymbol{\theta}), \boldsymbol{\Sigma}).
\end{equation}
Typically we just treat the visibilities as independent measurements ($\boldsymbol{\Sigma}$ is diagonal), and so we have a simple $\chi^2$ likelihood of
\begin{equation}
    \ln {\cal L}(\btheta) = - \frac{1}{2} \sum_i^N \frac{|\vd - \vm(\btheta)|^2}{\sigma_i^2}.
\end{equation}
Note that the likelihood is a scalar quantity. Next, we want to calculate the gradient of the likelihood with respect to the model parameters,
\begin{equation}
    \nabla \mathcal{L}
\end{equation}
which I assume will involve calculating the partial derivatives of $\vm$ with respect to the components $\btheta$. 

I think the idea is that we'll need to implement this math as \texttt{Theano} operations, and if there is something we need that \texttt{Theano} doesn't have, we'll need to implement this ourselves in \texttt{C++}.

\section{Toy sky model}
So, let's try to implement this in \texttt{Theano} starting with an analytic function for sky brightness and see if we can get \texttt{PyMC3} hooked up properly. Then we can move to more exciting sky brightness profiles. The simplest sky model I can think of is a Gaussian, since its Fourier transform is also a Gaussian. To keep things simple for now, we'll just consider a single channel. To make things more interesting (and hopefully catch edge-cases in the implementation), let's implement an elliptical shape, rotation, and phase-center offset. In this case, $\bbeta = \{a, \sigma_x, \sigma_y, \Omega, \delta_\alpha, \delta_\delta \}$, where $a$ is the peak amplitude of the Gaussian in the image domain and $\sigma_x$ and $\sigma_y$ are the semi-major axes of the Gaussian in the R.A. and Dec directions, respectively. We'll define the rotated coordinates according to Equations~\ref{eqn:rota}, \ref{eqn:rotd}, \ref{eqn:rotu}, and \ref{eqn:rotv}, where a positive $\Omega$ means that the Gaussian is rotated towards east of north (counter-clockwise in the sky plane). If the sky-plane model\footnote{Remember that $\alpha$, $\delta$, $\delta_\alpha$, and $\delta_\delta$ are in radians and $u$ and $v$ are in wavelengths.} is
\begin{equation}
    I(\alpha, \delta |\, \bbeta) = a \exp \left [ - \left \{  \frac{(\alpha^\prime - \delta_\alpha)^2}{2 \sigma_\alpha^2} + \frac{(\delta^\prime - \delta_\delta)^2}{2 \sigma_\delta^2} \right \} \right ]
\end{equation}
then the model visibility function is
\begin{equation}
    V(u, v |\, \bbeta) = a e^{- 2 \pi \imath (\delta_\alpha u^\prime + \delta_\alpha v^\prime)} \exp \left [ -2 (\sigma_\alpha^2 u^{\prime 2} + \sigma_\delta^2 v^{\prime 2}) \right ].
\end{equation}

In our implementation, we'll calculate $\vm$ as if we didn't know there were an analytic solution for $V$ (i.e., going through the FFT and convolutional sampling operations), and then we'll check to make sure it's the same in the end.

% \section{Extra stuff}
% Baseline-dependent phase errors may make this less true. See \citet{hezaveh13} for the proper way to bake these into the inference loop, if we so desire, but we will just ignore them for now.

% Crucially, and perhaps obviously, the maximum $u$ and $v$ in the dense visibility model must be larger than the largest $\{ u, v\}_i$ in the measurement set.

% \section{Modeling the radial intensity profile}
% How smooth is the radial profile, and can we use a Nuker profile to model it?


\acknowledgments

% \software{CASA \citep[v4.4;][]{mcmullin07}, IRAF \citep{tody86,tody93}, DiskJockey \citep{czekala15a}, RADMC-3D \citep{dullemond12}, emcee \citep{foreman-mackey13}, VARTOOLS \citep{hartman16}, Astropy \citep{astropy13}}

\bibliographystyle{yahapj.bst}
\bibliography{biblio.bib}


\end{document}
